%%% ---------------------------------------------------------------------------------
%%% Vorlage Abschlussarbeit (LaTeX)
%%% 
%%% V1   03/2017, Stefan Etschberger (HSA)
%%% V1.1 04/2021, rnw-hack für biblatex-run
%%% V2   05/2021, Titelblatt und Erweiterungen: Stefan Jansen (HSA)
%%% V2.1 05/2021, Trennung von R-Support und einfachem LaTeX: Phillip Heidegger (HSA)
%%% V2.2 01/2024, Anpassung an THA-Layout
%%% V3   01/2024, I18n
%%% V3.1 10/2024, Phillip Heideger, Online Version präferiert, Reparatur langes Inhaltsverzeichnis,
%%%               Erklärung Referenzen, blau für Refs & Links
%%% ---------------------------------------------------------------------------------
\documentclass[12pt,a4paper%
              ,oneside     % fuer PDF-Abgabe, bei Druck twoside
              ,titlepage
              ,DIV=13
              ,headinclude
              ,footinclude=false%
              ,cleardoublepage=empty%
              ,parskip=half,
              BCOR=0mm,
              ]{scrreprt}

\usepackage[utf8]{inputenc}
\usepackage[T1]{fontenc}

\usepackage[authorName={}
           ,authorEnrolmentNo={}
           ,authorStreet={}
           ,authorZip={}
           ,authorCity={}
           ,authorEMail={}
           ,authorPhone= {}
           ,authorSignaturePlace={}
           ,studyProgram={Mathematik}
           ,thesisType={Dokumentation}
           ,thesisTitle={NoSQL: Umsetzung der Semesteraufgabe mit Google Firestore}
           ,studyDegree=%
%                        {{Bachelor of Arts}}
%                        {{Bachelor of Engeneering}}
                        {{Bachelor of Science (B.\,Sc.)}}
%                        {{Master of Arts}}
%                        {{Master of Engeneering}}
%                        {{Master of Science (M.\,Sc.)}}
           ,faculty=% 
%                     {{Fakultät für \\ Angewandte \\ Geistes-  und  \\ Naturwissenschaften}}
%                     {{Fakultät für \\ Architektur und \\ Bauwesen}}
%                     {{Fakultät für \\ Elektrotechnik}}
%                     {{Fakultät für \\ Gestaltung}}
                      {{Fakultät für \\ Informatik}}
%                     {{Fakultät für \\ Maschinenbau und \\ Verfahrenstechnik}}, showDiesel=true
%                     {{Fakultät für \\ Wirtschaft}}
%           ,topicAssignment={\today}
%           ,submissionDate={\today}
%           ,defenseDate={\today}
%           ,nonDisclosure={false}
           ,supervisor={Prof.~Dr.~Frank N. Stein}
%           ,supervisorDeputy={Prof.~Dr.~Mario Huana}
%           ,language={en}
           ]{THA-docu}

% ====== mögliches Setup fürs PDFs =======
\hypersetup{
  colorlinks=true,
  allcolors = THAi-Blue, % oder THAred ??
  linktocpage  
}
% Siehe:
% https://ctan.net/macros/latex/contrib/hyperref/doc/hyperref-doc.pdf
% S. Kap. 5, S. 10 ff

% Ohne diese Zeile: Mit klickbaren links
% \hypersetup{draft}


% Literaturdatenbank (.bib-Datei) aus Citavi o.ä.
\bibliography{Literatur_docu}

\graphicspath{{Bilder/}}

\usepackage{caption}
\DeclareCaptionLabelFormat{something}{#2.#1.}
\captionsetup[lstlisting]{labelformat=something}

\begin{document}

% Sprachauswahl zum Umschalten innerhalb des Textes. 
% Alternativen: \thesisLanguage, ngerman, english
\selectlanguage{\thesisLanguage}

\pagenumbering{roman}
\setcounter{page}{1}

\THAtitlepage

\tableofcontents

%%% --------------------------------------------------
%%% Ab hier: Inhalt
%%% --------------------------------------------------

\setcounter{page}{1}
\pagenumbering{arabic}

\chapter{Ausgangssituation}
\chapter{No-SQL Datenbank Entscheidung}
\chapter{Firestore-Emulator Aufsetzung}
\chapter{Abfragensprache-Entscheidung}
\chapter{Typsicherheit mit TypeScript}
\chapter{Ansatz-Struktur-Entscheidung}
\chapter{Herausforderungen bei den Read-Abfragen}
Hier sind ein paar Einschränkungen, die wir bei der Umsetzung von Leseanfragen in Firestore hatten. 

\begin{enumerate}
    \item  Kein JOIN: Wir können in SQL mehrere Tabellen mit JOIN miteinander verknüpfen. Das geht nicht direkt in Firestore. Der Leseanfrage \textbf{d} könnte man in SQL mit Join von Angebot und Kurs über die Kurs-Nr. abfragen. Wir müssen zuerst alle Angebote in Firestore laden. Danach laden wir für jedes Angebot das dazugehörige Kursdokument aus der Kurs-Sammlung einzeln nach. Dadurch gibt es viele einzelne Leseoperationen. Das hätte für eine umfangreichere Datenbank zu höhere Zeitkosten geführt.

    \item Keine Aggregationen wie \textbf{COUNT, GROUP BY}: Firestore bietet keine Unterstützung für Aggregationen wie COUNT(*), AVG() oder GROUP BY. In Query \textbf{i} , in der alle Kurse mit mindestens zwei Teilnehmern gesucht werden, muss deshalb zunächst ein Zählerobjekt (teilnehmerCounter) im Code erstellt werden, das alle Teilnahmen durchläuft und pro Kurs-Angebot die Anzahl speichert. In einem relationalen Datenbank wäre dies eine einzige SQL-Zeile mit GROUP BY und HAVING COUNT(*) >= 2.
\end{enumerate}









\chapter{Herausforderungen bei den Update-Abfragen}
\chapter{Herausforderungen bei den Delete-Abfragen}
\chapter{Fazit}


\appendix

% Selbständigkeitserklärung
%\AuthorDeclaration


%\listoffigures % Abbildungsverzeichnis
%\listoftables % Tabellenverzeichnis

% --------------------------------------------------
% Bibliographie
% --------------------------------------------------
\renewcommand{\bibfont}{\footnotesize}
\printbibliography[title={Literaturverzeichnis}, 
                   heading=bibintoc]


% --------------------------------------------------
% Index
% --------------------------------------------------
{\setkomafont{section}{\Huge} % temporarily set chapter font
\printindex
}

\end{document}
